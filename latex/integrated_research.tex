\color{red}
The MISO abstraction symbolic execution framework as well as MISO-based PDAT
framework will be open-sourced. This will allow other researchers to perform
side channel analysis for their designs.  Runtime power estimates from high
level simulation and circuit level simulation as well as measurements from
prototyped chips will be made available as well. This data can be helpful for
wide variety of research.

This work will train several graduate and undergraduate students in a unique
and up-and-coming interdisciplinary area at the intersection of architecture,
tool development, and signal processing – this training will help add to a
unique workforce.  Undergraduate researchers working with PI Kumar have
recently published in premiere computer architecture conferences, including
ISCA, HPCA, and DAC. ISCA paper was covered widely by media. HPCA paper was
nominated for a Best Paper Award; DAC paper received significant media
coverage. Several undergraduate researchers have gone on to join graduate
studies at UIUC and elsewhere. The nature of the project will help us engage a
large number of undergraduate students in research.

Dr. Kumar has assisted in the organization of events such as HackIllinois and
Engineering Open House (EOH) that see enthusiastic participation from the local
community. He will have a booth on “processors that leak information” in the
next three EOHs to get high school students excited about intersection of
computer architecture and security.

Dr. Kumar was recently awarded the Ronald W Pratt Faculty Outstanding Teaching
Award and Stanley H Pierce Faculty Award at UIUC for, among other things,
innovative teaching of project-based courses in computer architecture and
system design. He also recently co-created a course on chip design and tapeout
where student groups perform tapeout of chips which then get manufactured by a
commercial foundry. Groups then do testing and bringup. The tapeout work
supporting the project will be done in context of this this course. Dr. Kumar
will blend in a discussion of side channel information leaking processors and
accelerators in his graduate computer architecture class. He will also list
side channel analysis and mitation related projects as project options in the
class, while making available the tools developed in this research.
\color{black}

