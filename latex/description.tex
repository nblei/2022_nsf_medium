\documentclass[11 pt]{article}

\usepackage[normalem]{ulem}
\usepackage{algorithm2e}
\usepackage{amsmath}
\usepackage{amssymb}
\usepackage{booktabs}
\usepackage{caption2}
\usepackage{cite}
\usepackage{color}
\usepackage{epsfig}
\usepackage{fancyhdr}
\usepackage{fullpage}
\usepackage{graphics}
\usepackage{graphicx}
\usepackage{hhline}
\usepackage{hyperref}
\usepackage{listings}
\usepackage{moreverb}
\usepackage{multirow}
\usepackage{pgfplots}
\usepackage{psfrag}
% \usepackage{subfigure}
\usepackage{tabu}
\usepackage{verbatim}
\usepackage{wrapfig}
% defines
\newcommand{\etc}{etc.\xspace}
\newcommand{\ie}{i.e.,\xspace}
\newcommand{\eg}{e.g.,\xspace}
\newcommand{\etal}{\emph{et~al.}\xspace}
%\newcommand{\comment}[1]{{\color{gray}[\textsf{#1}]}}
\newenvironment{packed_enum}{
  \begin{itemize}
  \setlength{\topsep}{0pt}
  \setlength{\itemsep}{1pt}
  \setlength{\parskip}{0pt}
  \setlength{\parsep}{0pt}
  \setlength{\parindent}{0pt}
}{\end{itemize}}

\long\def \ignoreme#1{}

% \usepackage[usenames]{color}
\usepackage{color}
\definecolor{gray}{rgb}{0.5,0.5,0.5}
\definecolor{dkgreen}{rgb}{0,0.6,0}
\definecolor{blue}{rgb}{0,0,1}
\definecolor{mauve}{rgb}{0.58,0,0.82}
%\ignoreme{
\newcommand{\alex}[1]{{\color{red}\bf [AO: #1]}}
\newcommand{\milos}[1]{{\color{red}\bf [MP: #1]}}
\newcommand{\alenka}[1]{{\color{red}\bf [AZ: #1]}}
\newcommand{\todo}[1]{{\color{red}\bf \ \\\noindent TODO: #1}}
%}
%\newcommand{\be}{\begin{equation}}
\newcommand{\ee}{\end{equation}}

\def\denseitems{
  \itemsep1pt plus1pt minus1pt
  \parsep0pt plus0pt
  \parskip0pt
  \topsep0pt
}
\leftmargini 1em
%  \leftmarginii-1em
%  \leftmarginiii-1em
%  \leftmarginvi-1em

\newcommand \co {\tt \small}

\setlength{\textheight}{9in}
\setlength{\textwidth}{6.5in}
\setlength{\topmargin}{0pt}
\setlength{\evensidemargin}{1pt}
\setlength{\oddsidemargin}{1pt}
\setlength{\headsep}{10pt}
\setlength{\parskip}{0ex}
\voffset -0.2in

\clubpenalty=10000
\widowpenalty=10000

\newcommand{\beq}{\begin{equation}}
\newcommand{\eeq}{\end{equation}}
\newcommand{\bsp}{\begin{split}}
\newcommand{\esp}{\end{split}}
\newcommand{\bal}{\begin{align*}}
\newcommand{\eal}{\end{align*}}
\newcommand{\bml}{\begin{multline}}
\newcommand{\eml}{\end{multline}}
\newcommand{\bi}{\begin{enumerate}}
\newcommand{\ei}{\end{enumerate}}
\newcommand{\bea}{\begin{eqnarray}}
\newcommand{\eea}{\end{eqnarray}}
\newcommand{\bc}{\begin{center}}
\newcommand{\ec}{\end{center}}
\newcommand{\denotes}{\stackrel{\triangle}{=}}
\newcommand{\la}{\langle}
\newcommand{\ra}{\rangle}
\newcommand{\nn}{\nonumber}
\newcommand{\bPsi}{{\mbox {\boldmath $\Psi$}}}
\newcommand{\her}{{\scriptscriptstyle H}}

\DeclareGraphicsExtensions{.png}

\newlength\figureheight
\newlength\figurewidth
\input{nsf-format}

%\renewcommand{\captionfont}{\small}
%\setcaptionmargin{0.05\textwidth}
%\setlength{\abovecaptionskip}{10pt}
%\newcommand{\ignore}[1]{}

\begin{document}

\begin{center}
{\Large \textbf{SHF:Medium:Making Analog Side Channels a First-Class Consideration in Architecture-Level Design}}
\end{center}

\section{Introduction}
Analog side-channels (power, electromagnetic, acoustic, etc.) have long been a potential source of attacks that circumvent traditional protections and security measures~\cite{217605,Backes:2010:ASA:1929820.1929847,Balasch2015DPABA,10.1007/978-3-319-66787-4_27,4812164,Chari:2002:TA:648255.752740,Genkin:2016:EKE:2976749.2978353}. This is especially a problem for cyber-physical systems (CPS) and Internet-of-Things (IoT) systems which often contain sensitive data, such as sensor data, login information for over-the-network management of the system and/or accessing back-end cloud infrastructure, and are often placed in publicly accessible  locations. For some side-channels, such as electromagnetic (EM) emanations, physical proximity can be leveraged to attack systems that are considered to be physically secure but are located near publicly accessible locations, e.g., in-wall ``smart building'' sensors, security cameras, etc~\cite{10.1007/978-3-662-48324-4_11,6766222,Camurati:2018:SCE:3243734.3243802,8574570}.
Many such attacks have been demonstrated over the past several decades, followed by countermeasures that prevent specific attacks by modifying the software that has been demonstrated to leak sensitive information. However, recent analog side channel attacks that were demonstrated by PIs Prvulovic and Zajic have shown that both attacks and mitigation are becoming increasingly dependent on microarchitectural behavior and potentially fragile to future microarchitectural changes~\cite{217605,Monjur21}.
Unfortunately, early-design tools, such as cycle-accurate (microarchitecture-level) simulators~\cite{Li:2009:MIP:1669112.1669172,Li:2011:CAM:2132325.2132479,509850,Ardestani:2013:EFM:2495252.2495480,Binkert:2011:GS:2024716.2024718,sesc,5982026,carlson2014aeohmcm}, do not model analog side channel signals, so these side channels can only be considered when they can be physically measured on already-fabricated chips. At that time, however, time-to-market concerns prevent introduction of overall design changes that would adjust the design tradeoffs in a more desirable direction. Additionally, most software developers have neither the know-how nor the equipment to assess their software’s potential vulnerability to analog side channels, so such considerations are typically either absent or qualitative/abstract when software is designed, giving first-mover advantage to attackers, and resulting in mitigation via localized patches, which themselves are becoming increasingly microarchitecture-dependent.

Ideally, the potential for information leakage through analog side channels and ``breaking'' existing software mitigation approaches would be considered in early stages of design for both hardware and software, guided by tools that can predict the impact a specific design has on analog side channels. This would be analogous to how performance and power consumption are predicted by cycle-accurate simulators, which allows the tradeoff between performance, power, and cost to be investigated at design time, years before the first prototype of that processor is fabricated ~\cite{Li:2009:MIP:1669112.1669172,Li:2011:CAM:2132325.2132479,509850,Ardestani:2013:EFM:2495252.2495480,Binkert:2011:GS:2024716.2024718,sesc,5982026,carlson2014aeohmcm} If such efficient-yet-highly-accurate simulation would exist for analog side channels, hardware designers and architects could include analog side-channel leakage among their design considerations~\cite{8416860,Althoff:2018:HII:3276539.3276601,Andrysco:2018:TVC:3243734.3243766,cryptoeprint:2018:808,Rane:2016:SPF:3241094.3241101,nayak2017hop}, compilers could use simulation models to optimize for reduced leakage~\cite{Liu:2015:GHS:2694344.2694385,Rane:2015:RCD:2831143.2831171,Gorman:2017:AON:3123939.3123973}, software designers could detect and mitigate information leakage problems for security-sensitive applications~\cite{Wichelmann:2018:MFF:3274694.3274741,Chen:2017:PDS:3133956.3134058,Wu:2018:ETS:3213846.3213851}, etc.

While there are some tools and metrics to quantify analog side-channel leakages~\cite{Demme:2013:FOM:2485922.2485970,Callan:2014:PMM:2742155.2742179,McCann:2017:TPT:3241189.3241207,Barenghi:2018:SSS:3195970.3196112, yilmaz17tifs}, they are limited due to following reasons: \textit{First}, they are mainly focused on developing metrics to estimate the information leakage itself, i.e., mutual information between the signal and the program secrets, rather than modeling the actual analog signal. Relying only on these metrics rather than analyzing the actual signal may not be sufficient, as these metrics inherently make assumptions about the aspects of the signal the attacker may exploit, i.e., they may not reveal \emph{all} of the information the signal may contain. \textit{Second}, most existing methods only model the system at \textit{architecture-level}, i.e., associating a (leakage) value to individual instructions based on the ISA, and ignore the micro-architecture activities such as pipeline stages, stall cycles, etc. on the signal. As we have demonstrated in our seminal paper on analog side channel modeling~\cite{Nader2020} (these findings will be briefly be summarized in Section~\ref{sec:proposed}), this can lead to significant inaccuracy, mainly because the model, by staying at the ISA-level, neglects to account for how that instruction interacts with other instructions and the underlying hardware.  \textit{Third}, by neglecting the impact of micro-architecture, these methods implicitly assume that the entire hardware design is a \textit{single source} of the combined signal. Such an assumption can lead to large inaccuracies for some side channels because contributions of different microarchitectural blocks (e.g., cache, register-file, etc.) may experience different propagation (e.g. attenuation, low-pass filtering, etc.) to where the overall signal is being measured.

The PIs first attempt to address these challenges and develop microarchitecture-level analog side-channel model was presented in \cite{Nader2020}. This work was a proof-of-concept study, showing that efficient yet highly accurate estimation of analog side channel signals is feasible, not only for individual instructions but also for sequences of instructions (when the goal is to assess and improve leakage from a particular piece of code on a set of hardware platforms). This model is also the first to asses leakage from a particular part of the system (when the goal is to make the design less ``leaky''), while maintaining the performance advantages of a cycle-accurate simulation relative to gate-level or physics-based models. While the proposed model was good enough to demonstrate feasibility, and can serve as an excellent starting point for our proposed work (as will be discussed in Section~\ref{sec:proposed}), it used a number of simplifications that limit its ability to model more complex processors and limit its accuracy.

\subsection{Proposed Research Work}
To address these issues and allow analog side channels to be treated as first-class design considerations, we propose to develop methods that would allow microarchitecture-level simulation to generate highly accurate analog side channel signals, even for sophisticated processors (deep pipelines, superscalar, out-of-order). Specifically, the proposed research consists of developing four main thrusts:

\begin{enumerate}
\item Methods that allow microarchitecture-level simulation to produce highly accurate analog side channel signals, along with per-block and per-code-region breakdowns and attribution to help computer architects and software developers assess the impacts of microarchitectural and software changes on the tradeoff between performance, power, and side channel leakage. Our overall approach to this will be to generate detailed event timestaps and use them to combine per-event signal snippets into the overall side channel signal.
\item Methods for efficiently performing circuit-level simulation of caches, functional units, and other building blocks of the processor, to provide per-event timelines for voltage, current, and possibly other metrics that are needed to construct signal snippets for various side channels, along with other useful information such as how much the values of the metrics depend on data values.
\item Methods that produce per-event signal snippets for various side channels, by 1) transforming the outputs of circuit-level simulation, 2) decomposing measured signals, or 3) synergistically combining both methods, and methods that estimate per-block scaling factors for adding these snippets to the overall analog side channel signals.
\item Methods that validate simulated signals against measurements, and to adjust scaling factors, signal snippets, and the simulator's timing parameters to calibrate snippets, weights, and timing parameters to a real processor.
\end{enumerate}

 %If successful, our work will demonstrate the feasibility of modeling analog side-channels at the microarchitectural level and provide proof-of-concept integration of such modeling into a cycle-accurate simulator. This will allow analog side channels to become a first-class consideration, along with performance and power, in processor designs, allowing computer architects to avoid introducing significant new vulnerabilities and ``breaking'' existing software mitigation, and possibly even to reduce leakage and/or enable new mitigation. It would also allow programmers and even compilers to include analog side channel considerations in their tradeoff space during design and/or optimization.
\subsection{Broader Impacts}
We expect that our results will help the inclusion of analog side channels among early design considerations and will help reduce the cost of side-channel resistant designs by addressing side-channel-related problems early in the design process, when side-channel resilience may be improved (or preserved) with little or no sacrifice in performance, power, cost, weight, etc. The proposed work is inherently interdisciplinary, combining expertise in computer architecture and circuits, hardware security, electromagnetic, and signal processing. Thus this research has the potential to improve the state of the art and have broader impacts in all these areas. Also, participating students will be working in a truly multidisciplinary context, which will broaden their expertise in ways otherwise not possible as well as {\bf broaden participation in computing}. The proposal also includes 1) developing an interactive demonstrator for the general public, to educate and raise awareness about several key cyber-security concepts and issues, 2) visits and activities in local schools to improve K-12 education and participation of women and minorities in STEM, and 3) course and curriculum development activities at the undergraduate and graduate level.
\section{Related Work}
There is a large body of work focused on preventing particular side-channel attacks, e.g., ~\cite{Backes:2010:ASA:1929820.1929847,Nazari:2017:EED:3079856.3080223,Demme:2013:FOM:2485922.2485970,Han:2017:WMB:3133956.3134081,Liu:2016:CET:2976749.2978299,6987331,He:2017:SYC:3123939.3124546,217605,Monjur21}, either by removing the tie between sensitive information and the side-channel signal, or by
trying to make the signal more difficult to measure. However, such work mostly focuses on preventing a particular side
channel attack in a very specific piece of code and are less focused about the fundamental relationship between the hardware, software, and the side-channel signal.

Strategies for quantifying potential side channel exposure at the micro-architectural and architectural levels are still
an open problem. Existing work proposed different methods and/or metrics to estimate the leakage either for a specific type of side-channel (e.g., cache, power, EM, etc.) or alternatively, as a generic framework to estimate the overall leakage for any given side-channel.

Side-Channel Vulnerability Factor (SVF)~\cite{Demme:2013:FOM:2485922.2485970} measures how the side-channel signal correlates with
high-level execution patterns (e.g., program phase transitions).
While this metric allows overall assessment of the
``leakiness'' of a particular system and application over a
given side-channel, it provides limited insight to 1) computer
architects about which architectural and microarchitectural
features are the strongest leakers, and to 2) software developers
about how to reduce the side-channel leakiness of their
code.

To address these limitations, Signal Available to Attacker (SAVAT) method~\cite{Callan:2014:PMM:2742155.2742179} was proposed. SAVAT measures the side-channel signal (particularly EM and power from laptops) created by a specific single-instruction difference
in program execution, i.e., the amount of signal made available
to a potential attacker who wishes to decide whether the
program has executed instruction/event A or instruction/event B. These measurements can be used
to determine the potential for information leakage when
execution of individual instructions or even sections of code
depend on sensitive information. Unfortunately, SAVAT only models the system at ISA-level and ignores the underlying relation of each instruction to the hardware or other instructions in the sequence. Extensions of SAVAT that include modeling instructions overlapping in pipeline have been proposed in \cite{baki17}, \cite{Baki_2020a}.

Similar to SAVAT, McCann \textit{et al.}~\cite{McCann:2017:TPT:3241189.3241207} proposed a modeling technique capable of
producing a leakage metric at instruction-level for power (and/or
EM) side-channel signals on ARM M0/M4 cores. To estimate the leakage for individual instructions, the proposed method  only requires knowledge about different characteristics of the system at ISA-level such as data-dependent effects of neighboring
instructions in a sequence, register effects, bit-flips, etc. Similar to SAVAT, while the method proposed by McCann \textit{et al.}~\cite{McCann:2017:TPT:3241189.3241207} provides interesting insights about possible sources of leakage, it also ignores the effects of micro-architecture events such as cache miss, branch miss-prediction, etc. on the signal.

The method proposed by Barenghi and Pelosi~\cite{Barenghi:2018:SSS:3195970.3196112} calculates the leakage for individual instructions by measuring the power consumption between two consecutive cycles and employs the Pearson correlation coefficient between the two measurements. To calculate the leakage, in addition to leveraging the ISA-level information, pipeline model was also used. However, the framework did not consider any micro-architecture events, nor pipeline stalls and only accounts the number of cycles that takes for each instruction to execute. It also did not model the individual effect of each stage on the others and the overall signal.

Another approach to quantifying side channel leakage is to use information theory and estimate capacity of analog side channels. Millen was the first to establish a connection between Shannon's information theory and information flow models in computer systems~\cite{Millen87}
and calculated the capacity (maximum possible data rate) of such a side-channel.
However, that model assumes a synchronous channel (where information is transmitted at a constant rate that is known to the receiver),
and this is not a realistic assumption for analog side-channels in computer systems, where the timing of execution in the spy program varies
due to a number of hardware features (e.g., pipeline stalls, dynamic scheduling of instructions, cache hits and misses, branch prediction, etc.).
Additionally, analog side-channels often include insertion, deletion, and errors, e.g., interrupts and other system activities often insert activity into the timeline of the spy program's execution. There are many papers that discuss bounds on the capacity of channels corrupted by synchronization errors
~\cite{Wang05}, \cite{Anderson98}, \cite{Crespi13}, \cite{Davey01}, \cite{Ramachandran11}, \cite{Kirsch10}, \cite{Hu10}, bounds on the capacity of channels corrupted with synchronization and substitution errors \cite{Verdu10}, \cite{Duman13}, \cite{Tarokh12}, or bounds on the capacity when codewords have variable length but no errors in the channel \cite{Verdu10}, \cite{Shannon:InfoPaper}, none of them provides the answer to how much information is ``transmitted'' by execution of particular sequence of instructions that do not have equal timing and are transmitted through erroneous channel.
The first attempts to answer this question were presented in \cite{baki18icassp,baki18milcom}, where covert channels are generated, and upper and lower leakage capacities were derived. In \cite{baki17}, a side channel leakage capacity is derived for a discrete memoryless channel where it was assumed that each transmitted quantum of information (i.e., instruction in the code) is mutually independent but do not have equal length. Although all these papers make an important step toward assessing information leakage from side-channels, they fall short of considering the relationship among sequence of instructions, which is a result of program functionality as well as a processor pipeline depth, which impacts how much signal energy will be emanated.
In \cite{Baki_2020a}, side-channel information capacity created by execution of series of instructions (e.g., a function, a procedure, or a program) in a processor is derived. To model dependence among program instructions in a code, we use Markov Source model, which includes the dependencies that exist in instruction sequence since each program code is written systematically to perform a specific task. The presented  framework  considers processors as the transmitters of a communication system with multiple antennas. The antennas correspond to different pipeline stages of any processor. Moreover, inputs of the transmitter show dependency based on a Markov model which reflects the practicality of a program. Using this setup, we have obtained the channel capacity of a communication system which represents the severity of the side channels. However, none of these approaches provide enough details to model analog side channels taking into consideration microarchitecture details.

Another body of work related to this proposal are the cycle-accurate models/tools to simulate power and/or microarchitecture~\cite{Li:2009:MIP:1669112.1669172,Li:2011:CAM:2132325.2132479,509850,Ardestani:2013:EFM:2495252.2495480,Binkert:2011:GS:2024716.2024718,sesc,5982026,carlson2014aeohmcm}. While these models can accurately model the power consumption at each cycle, they are different from this work and hence may not be a proper tool for simulating analog side-channel signals for two main reasons: \textit{First}, while these methods do consider the activity factor to calculate power, they often treat all the bit-flips equally. However, as shown in \cite{Nader2020}, depending on the design, not all flips equally contribute to the overall signal. Ignoring this fact can lead to inaccurate modeling. \textit{Second}, depending on the architecture, different stages might have different effect on each other and the overall signal. Without properly modeling these effects, the overall signal can not be modeled.

Also related to this work are work on leveraging EM signals for program profiling, tracking, and analysis~\cite{7783762,Gorman:2017:AON:3123939.3123973,Elvan2021,Zop, Zop2,Moumita2018,Moumita2022}. Spectral Profiling methods~\cite{7783762,Zop,Zop2,Elvan2021} tracks program execution at different granularity from loops \cite{7783762}, to basic blocks \cite{Zop,Zop2} and individual instructions \cite{Elvan2021}. \textsc{EMProf} profiles memory \cite{Moumita2018} and \textsc{Primer} profiles interrupts \cite{Moumita2022}, both leveraging electromagnetic (EM) emanations from devices. By continuously analyzing these EM emanations, \textsc{EMProf} identifies where in the signal's timeline each period of stalling begins and ends, allowing it to both identify the memory events that affect performance the most (LLC misses) and measure the actual performance impact of each such event (or overlapping group of events). Because \textsc{EMProf} is completely external to the profiled system, it does not change the behavior of the profiled system in any way, and requires no hardware support, no memory or other resources, and no instrumentation on the profiled system. Similarly, \textsc{Primer} analyzes the device's external EM side-channel signal in real-time, without any interference with the device's program execution, while providing a detailed analysis of not only the overall overhead created by interrupts, but also their distribution over time (i.e., exact occurrence of interrupts in program execution time-line). As the CPU follows a generic procedure to handle such asynchronous system events, our approach can be generalized and applied to all non-deterministic interrupts across multiple platforms. More details on both methods are presented in following sections. All these methods are complimentary to this proposal and combined with proposed modeling tool can enhance leakage estimation, compiler development, etc.

However, none of them address the need for developing microarchitecture-level analog side-channel tool such that allows for integration of such modeling into a cycle-accurate simulator. This, in turn, would allow analog side channels to become a first-class consideration, along with performance and power, in processor designs, allowing computer architects to avoid introducing significant new vulnerabilities and ``breaking'' existing software mitigation, and possibly even to reduce leakage and/or enable new mitigation.

\section{Proposed Research}
\label{sec:proposed}

\begin{wrapfigure}{r}{0.4\textwidth}
	\centering
	\includegraphics[width=0.4\columnwidth,trim={0in 2.65in 6.89in 0in},clip]{figure/overview.pdf}
	\caption{Overview of the proposed method.}
	\label{fig:overview}
\end{wrapfigure}

Figure~\ref{fig:overview} shows our overall proposed simulation approach. Most parts of this approach are within the orange-background ``Training'' oval in Figure~\ref{fig:overview}, which indicates that they are only needed for training, before the cycle-accurate simulation that constructs the overall simulated side channel signal begins. We propose to implement this overall signal construction efficiently by first preparing signal snippets for each type of event in each of the microarchitectural buling blocks within the processor, along with per-building-block weights that specify how to scale the contribution of each block when combining it into the overall side channle signal. Given these snippets and weights, we can obtain timestamps of individual events through cycle-accurate simulation and construct the overall signal by adding the appropriate snippet to the signal at each point in the signal's timeline where the corresponding event occurs. Our training (the ``Training'' oval in Figure~\ref{fig:overview}) constructs the signal snippets and their estimates weights through circuit-level simulation of individual events in individual building blocks, through decomposition of measured signals from an actual processor, or through a combination of these methods. Finally, simulation-generated signal can be calidated against measured signals and, if callibration agains that real processor is desired, discrepancies found during validation can be used to improve the snippets and weights in order to bring the future simulated results into better agreement with the measured signals.

To implement this approach, our proposed research consists of four primary research thrusts:

\begin{enumerate}
\item Finding how to efficiently generate detailed event timestaps and use them to combine signal snippets into the overall side channel signal, along with per-block and per-code-region breakdowns and attribution to help computer architects and software developers assess the impacts of microarchitectural and software changes on the tradeoff between performance, power, and side channel leakage.
\item Finding how to efficiently perform circuit-level simulation of caches, functional units, and other building blocks of the processor, to provide per-event timelines for voltage, current, and possibly other metrics that are needed to construct signal snippets for various side channels. along with other useful information such as how much the values of the metrics depend on data values.
\item Finding methods that produce per-event signal snippets for varous side channels, by 1) transforming the outputs of circuit-level simulation, 2) decomposing measured signals, or 3) synergistically combining both methods, and methods that estimate per-block weights.
\item Developing methods to validate simulated signals against measurements, and to adjust weights, signal snippets, and the simulator's timing parameters to calibrate snippets, weights, and timing parameters to a real processor.
\end{enumerate}

In the rest of this section, we provide more detail on each of these research thrusts, along with preliminary results that demonstrate the feasibility of the proposed approaches and provide an excellent starting point for the proposed research in each thrust.

\subsection{Thrust 1: Construction of Side-Channel Signals Through Cycle-Accurate Simulation}

With the scaled per-event signal snippets $S_b(t) , \dot \alpha_b$ from training, we can construct the side channel signal that corresponds to longer-term program execution. This signal construction uses a microarchitecture-level cycle-accurate simulation to determine the times at which each event occurr in each of the processor's block-level parts (ALU, decoder, reservation stations, cache, etc.), and these per-block event timestamps are then used to combine per-event signal snippets into an overall simulated signal. If validation against a real processor is desired, the simulation-generate and measured side channel signals can be compared, and the results of this comparison can be used to improve the estimation of per-block weights, or even (not shown in Figure~\ref{fig:overview})  to choose another circuit-level model that better matches the actual hardware of the processor.

One of the key research problems for this thrust will be to identify the tradeoff between the accuracy of the generated signal and the level of detai in the cyle-accurate simulation. Less detail, e.g. for an L1 data cache we may lump all events (hit, lookup with miss, fill following a miss) into a single ``cache action''  category that is represented by a single snippet. This would still allow the generated signal to capture the impact of the everall timeline of cache activity, e.g. having more prominent signals when executing code code regions with a lot of cache accesses, having low-signal-activity periods when the cache is not used (due to a processor stall between a miss and the corresponding fill), etc. However, these events would need to be differentiated to account for differnces between reads and writes, for lookups with a miss having to initiate an access in the next-level cache while hits to not, for fills writing an entire line while a write hit writes only a small part of the line, etc.

A similar tradeoff will be explored with respect to the impact of data values. We will start with data-value-agnostic simulation, which can still account for e.g. attackers being able to determine which part of the code is being executed, e.g. due to secret-dependent control flow. We will then add ability to obtain data-value statistics in cycle-accurate simulation,
and leverage the signal envelopes produced by our symbolic circuit-level simulation, to identify events (and blocks) whose signal snippets are the most affected by data values. For such blocks/events we will identify the nature of data-value dependency and construct data-value-dependent snippets, stansformations, and weights to account for data-value dependencies in the signal. We will do this selectively (not for all events) because there is a tradeoff between accuracy and training/simulation time, so a given amoung of accuracy improvement will come at a lower cost in training/simulation time is only the most data-value-dependent signal snippets are refined in this way.

\subsubsection{Preliminary Results for Thrust 1}

Our preliminary results for this thrust are based on EMSim~\cite{Nader2020}, a first-of-its-kind cycle-accurate simulator that generates practically relevant EM side channel signals for program execution on a canonical 5-stage pipelined processor. This simulator,
implemented by PI Prvulovic, Co-PI Zajic, and several PhD students they have co-advised, demonstrates the feasibility of using event timings from microarchitectural simulation together with signals snippets to generate overall side channel signals.

In EMSim, each pipeline stage of the processor is modeled as a single block, and each instruction is treated as a separate event for each stage. The signal snippets all use the same shape, which is a mathematical function. The results we show here use an sinusoid with rapidly declining (exponential) magnitude as the snippet shape. However, different events use snippet amplitudes (i.e. scaling factors), which are estimated using linear regression from signals that are measured on an FPGA-based impementation of this processor. Note that, although a more detailed RTL model of this processer was available,
we only used the microarchitectural model in our simulation, while the RTL was only used to actually implement the processor, and the scaling factors are then empirically obtained by running microbenchmarks on that real processor, measuring signals, and using linear regression to determine the  scaling factors that result in the best fit between the simulated and measured signals for those microbenchmarks..

\begin{wrapfigure}{l}{0.4\textwidth}
	\centering
	\includegraphics[width=0.4\columnwidth,clip]{figure/bench.pdf}
	\caption{A comparison between the signal generated by a real hardware (top) and the simulated signal (bottom) in EMSim.}
	\label{fig:bench}
\end{wrapfigure}
The scaled snippets are then combined into the overall simulated side channel signal for previously unseen program code (not used to obtain scaling factor). For this, we use a microarchitecture-level simulator to model the pipeline of the processor. In each cycle,
for each stage of the processor this simulator identifies which instruction is in that stage in that cycle, i.e. which event occurs in each of the ``building blocks'' of the processor, and
the ``snippet'' with the block-and-event-dependent magnitude is added to the overall signal at that point in time. The resulting signal for one such simulation is the bottom signal in Figure~\ref{fig:bench}, while a measured signal for the same program sun (same code and data values) is shown above it (also in Figure~\ref{fig:bench}). We observe that, alstohugh the signals have some differences, most of the prominent features from the measured signal are also present in the simulator-generated signal.

In the same work (EMSim~\cite{Nader2020}), we have also demonstrated the importace of modeling timing and microarchitectural behaviors, i.e. the need for using a cycle-accurate simulation of the microarchitecure rather than a simpler emulator of the instruction stream.

\begin{wrapfigure}{r}{0.4\textwidth}
	\centering
	\includegraphics[width=0.4\columnwidth,clip]{figure/stall2.pdf}
	\caption{Effect of stalls on the signal. The actual signal shown in green, while simulated signals are shown in black when modeling pipeline stalls (top) and not modeling it (bottom).}
	\label{fig:stall}
\end{wrapfigure}
To illustrate the need to account for microarchtectural events, in Figure!\ref{fig:stall} we show the effect of a pipeline stall (for clarity, we cause a stall by increasing the latency of a MUL instruction to eight cycles, where the next instruction uses the result of that MUL) on the signal's timeline. The top part in Figure~\ref{fig:stall} is the simulated signal that models the pipeline stall, along with the measured signal for the same part of the program execution. However, it pipeline stalls are not accounted for (in this case, by ignoring the effects of the dependence between the MUL and the next instruction), the timeline of the simulated signal changes significantly (bottom part of Figure~\ref{fig:stall}) and has no longer has an excellent match to the measured signal. Similar effects are observed when the simulation ignores other microarchitectural events and their timing, e.g. if caches are not modeled (all load/store treated as hits), if branch prediction is not modeled (no branch mispredictions), etc.

\begin{wrapfigure}{l}{0.4\textwidth}
	\centering
	\includegraphics[width=0.4\columnwidth,clip]{figure/mise2.pdf}
	\caption{Effect of misprediction (right) on the signal. It causes two instructions being flushed from the pipeline and hence affect the signal in those cycles.}
	\label{fig:mis}
\end{wrapfigure}
In addition to its impact on timing, we observed that a branch misprediction also has noticeable impact on the shape of the side-channel signal. In the 5-stage processor that was used in EMSim, a branch misprediction causes a flush of the first two pipeline stages, which not only adds two cycles to the execution but also changes which instructions are in which stage of the pipeline for the next few cycles, as shown in Figure~\ref{fig:mis} with real (measured) signals. Therefore, if the simulated signal is put together without modeling branch prediction (and accounting for mispredictions), the signal differs significantly. This effect would be even more pronounced for a more sophisticated processor (more instructions flushed and more change in overlap among activity in different stages/blocks).

Overall, even though EMSim had used the a simple mathematical function as the snippet shape for all events, the side channel signal it constructs matches the measured EM side channel signal relatively well. To evaluate EMSim, we had used programs that were specifically generated to create various combinations of instructions in different pipeline stages, resulting in 34 programs that execute for about 9,000 cycles each. We used \textit{normalized cross-correlation} as our accuracy metric.

Overall, we found that EMSim achieves 94.1\% accuracy in simulating the EM side-channel signals. While this is sufficient for many purposes, e.g. assessing vulnerability to simple power attacks (SPA) and simple EM attacks (SEMA) that extract secrets by differentiating between executing different parts of the program code (e.g. a secret-dependent if-then-else) and/or differential attacks (DPA, DEMA) that depend on large differences in values that are operated on, more accuracy is likely to be needed for simulated signals to be used to assess vulnerability to more sophisticated attacks that differentiate between data values that are similar to each other. We expect that the proposed use of per-event signal snippets from circuit-level simulation, with additional data-value-dependent considerations, will enable such uses of our simulation-generated signals. Additionally, our proposed work will enable signals to be generated not only for the (simplest) pipelined processors, but also to the much mode complex processors (deep pipelines, superscalar, out-of-order, etc.), by extending cycle-accurate simulators such as GEM5~\cite{gem5}.

\subsection{Thrust 2: Construction of Per-Event Current/Voltage Timelines via Circuit-Level Simulation}

Even when a full circuit-level model for an entire processor is available, circuit-level simulation is not efficient enough to go through numerous program runs, with billions of processor cycles in each run, that would be needed to provide computer architects and software developers with actionable insights about the impact of microarchitectural and/or software changes on the tradeoff between performance, energy, and side-channel vulnerability. For example, PI Kumar's symbolic execution framework~\cite{cherupalli2017} currently uses detailed gate-level models of the microprocessor hardware, resulting in high-fidelity reproduction of metrics that are needed to predict the microprocessor's EM and power side-channel signals, but this  at a high computational cost when modeling entire-program execution -- an FFT implementation for a 16-bit MSP430 DSP takes over 3 hours, and this time would become several times longer for more complex processors and orders of magnitude longer for realistic runs of larger applications.

Instead, our proposed approach raises the abstraction level of the whole-program simulation to the microarchitecture level (i.e. cycle-accurate simulation), and will use circuit-level simulation only during training, separately for each block of the processor, e.g. decoder, adder, multiplier, instruction scheduler, cache, etc. to produce, for each kind of activity in a block, how currents and voltages in that block change over time (i.e. $I_b(t)$ and $V_b(t)$) during that activity. Note that each of these circuit-level simulations is very brief, as it only needs to account for activity that corresponds to a single event (e.g. for a cache the events would be a lookup with a read-hit, lookup for a read-miss, fill on a read-miss, and read completion after a miss and fill, and the corresponding events for a write), so each of these circuit-level simulations models only a few processor cycles, albeit with a great level of detail.

To improve accuracy of our overall signals, we will leverage the ability of the existing circuit-level simulator to evaluate circuit activity for symbolic data.  As such, for a given time period, the current, voltage, etc. signals are each best modeled as a time-chaging distribution, or as an  ``envelope'' that shows the possible range of values at each point in in time, rather than a single curve. This presents us with a number of research challenges: how to express these distributions in the way that aids accuracy of the overall signals produced by cycle-accurate simulation, how to prevent path explosion when modeling blocks that may have significant changes in activity depending on the data (e.g. a symbolic address causing a hit or a miss in a cache), how to express aggregate data-dependent amounts as simplified symbolic expressions, to allow microarchitecture-level simulation to account for significant changes in aggregate activity within a block without resorting to modeling the block at a lower lever of abstraction, etc.

To help designers of highly secure microprocessors, i.e. processors which should have minimal information leakage via side-channels for {\tt any}program, we will also investigate a complementary approach, Property Driven Automatic Transformation (PDAT)~\cite{bleier21}, which analyzes net switching activity within a circuit for any inputs and performs model checking that allows processor designers to choose (or redesign) specific building blocks in a way that minimizes data-dependent changes in activity that would leak information. In essence, rather than accounting for data-dependent activity and expressing it in the signals provided to cycle-accurate simulation, this approach changes the hardware of the block to minimize such data-dependent activity, likely eliminating the need to account for data-dependent activity for that block.

Finally, while we expect that overall current and voltage will allow our side-channel-specific transformations to produce highly accurate signal snippets, more entailed outputs, e.g. separate $I_b(t)$ and $V_b(t)$ for different kinds of circuit elements, e.g. transistors, interconnect wires, power delivery wires, etc. may be needed to further improve accuracy, and we will investigate both the need for such refinements, and how to implement them efficiently. For example, transistors that have the same geometry may have similar impact on side-channel-relevant properties (current, voltage, etc.), and thus may be represented in aggregate, but activity of larger transistors may need to be aggregated separately from smaller ones, to allow different scaling factors to be used for their activity when generating side channel snippets and/or accounting for data-dependent changes in the side channel signal.

\subsubsection{Preliminary Results for Thrust 2}

\begin{wrapfigure}{l}{0.4\linewidth}
\includegraphics[width=\linewidth]{./figure/Bespoke_Flow_Fig.pdf}
\caption{\small
    Hardware-software co-analysis used to generate bespoke microprocessors.}
\label{fig:bespoke}
\end{wrapfigure}
Co-PI Kumar's prior efforts to enable software developer to analyze their software for analog side channels have resulted in a framework which performs symbolic execution of a program on a detailed (gate-level) model of a microprocessor~\cite{cherupalli2017}.
This framework was originally developed in order to automate design of `bespoke microprocessors' --- microprocessors designed to execute a single application binary.
The flowchart for this use case is in shown in Figure~\ref{fig:bespoke}, and the symbolic execution of the program on the hardware occurs in the `Gate Activity Analysis' stage.

This framework accounts for how activity within the processor is affected by program inputs (e.g. sensor inputs in an IoT system), both for data-dependent (data values) and instruction-dependent (which instructions end up being executed) activity within the circuitry of the processor. This symbolic execution framework can identify whether any possible execution of a program leads to significant changes in gate-level activity and, consequently, side-channel leakage, allowing programmers to to add or remove often-expensive software mitigation techniques depending on which mitigations are actually needed. This framework has shown applicability for security applications~\cite{cherupalli20172}, including
taint-tracking for information flow security to locate information policy
violations, even violations which result from covert timing side-channels.

This prior work provides us with an excellent starting point for the proposed symbolic circuit-level simulation. However, as pointed out earlier, circuit-level simulation of an entire processor throughout the execution of a program is exceedingly slow compared to state-of-the-art microarchitecture-level (cycle-accurate) simulation. Our proposed approach will synergistically combine the advantages of symbolic circuit-level simulation and microarchitecture-level simulation, using circuit-level simulation only duringtrainingto generate accurate per-event signal snippets, and using the (orders of magnitude faster) microarchitecture-level simulation to model the entire microprocessor for an entire program execution and generate a side channel signal by combining pre-made snippets.

Additionally, which gate-level switching activity can be a good proxy for side channel information leakage, PI Prvulovic and Co-PI Zajic's past work~\cite{Nader2020} has shown that the mapping from gate-level activity to side channel signals tends to be proportional only for gates within a relatively homogenous block (e.g. a cache, a multiplier, etc.), but the scaling factors for different blocks can differ significantly. Thus our proposed approach is to account for these unequal transformations on a block-by-block basis, accounting for the specifics of how each side channel (EM, power, etc.) is affected by gate-level events, to produce {\em side channel signals} (rather than proxy metrics) that can be compared to and callibrated against measured signals for varous side channels/

\subsection{Thrust 3: Per-Event Side Channel Snippet Construction and Scaling}
\label{sec:proposed-transform}

This research thrust will explore 1) how to transform the results of the circuit-level simulation into a signal snippet that represents the shape of the side channel signal that is contributed to the overall side channel signal by an occurrence of that event, 2) how to estimate the magnitude and (at least to some extent) the shape of these side channel signal snippets by decomposing measured signals, and 3) how to synergistically combine the side channel snippets obtained from circuit-level simulation and from signal decomposition.

Specifically, after the circuit-level simulator produces $I_b(t)$ and $V_b(t)$ signals for a specific event that happens in a specific block, we need to transformed this into side-channel signal snippet $S_b(t)$. This step accounts for the impact that current and voltage (and their changes over time) have on that side-channel. For example, for the power side channel the transformed signal would be the product of the current and voltage signals, for the EM side channel the transformed signals would be using Maxewell' equations to produce transformed signals that are proportional to the first derivative of the curent and voltage signals, etc.

Snippets $S_b(t)$ can also be obtained from measurements of side channel signals on a real processor. This is needed, for example, when a circuit-level model of some part of the processor is not available, or to correct (or calibrate) the scale of snippets from circuit-level simulations that have systematic biases, e.g. due to imprecisions in transistor models or lack of accounting for some electrical, electromagnetic, or quantum effects during circuit-level simulations. Our initial approach to obtaining per-event snippets from measured data will
be to collect side channel signals as the processor executes various microbenchmarks that are designed to create specific of events in specific parts of the processor, allowing the measured signals to be decomposed into signal snippets that correspond to contributions of individual events in individual units (e.g. using SVD).

Finally, we note that the two methods of obtaining per-event signal snippets are highly complementary. Specifically, because of measurements noise and bandwidth limitations, these measurement-derived snippets tend to be more accurate in terms of the overall contribution of the event to the signal (e.g. the average signal magnitude contributed by the event) over its ``life span'' (i.e. the few cycles during which the single occurrence of the event affects the signal), and less accurate in terms of the shape, i.e. change-over-time of the signal snippet during the event's ``life span''. Conversely, circuit-level models are likely to produce very accurate shapes -- if N identical-in-size transistors toggle, the current draw is proportional to N -- but the magnitude of these shapes may not be scaled correctly if the transistor parameters do not match the actual ones precisely. This means that, when both circuit-level models and empirical measurements are available, the model-generated signal snippets can be scaled to match the average magnitude of measurement-derived signals. This combining, along with other activities that involve scaling and callibration of simulated against measured side-channel signals, is part of the next research task.

As already noted in Section~\ref{sec:proposed-transform}, per-event snippets generated by from circuit-level simulations contain a lot of detail and metadata (e.g. the ``envelope'' for the signal that shows how much the signal snippet can change due to different data values), but may have systematic biases due to imperfect modeling of device physics, device parameters, or the circuit, and these biases are likely to mostly affect the scale of the snippet rather than its shape. Conversely, obtained by decomposing measured signals have much more accuracy in their overall magnitude (scale) than they have in their exact shape. To synergistically combine these methods, our initial approach will be to estimate the per-block scaling factors (weights) $\alpha_b$, and we will refine this as needed to improve accuracy further (e.g. using functions other than simple multiplication with a constant scaling factor).

The second reason for using weights (or other ``scaling'' functions) is to account for side-channel specific differences in attenuation of signals from different blocks. Specifically, for the EM side channel, the EM signal contribution from one block may experience more propagation loss than signals from another block, depending on the materials each signal propagates through and/or the position of the antenna or probe. These differences in propagation can also be accounted for using the per-block scaling factors $\alpha_b$, which can be determined using either measured signals or simulation-based propagation models.

Finally, we will account for data-dependent activity by accounting for how the signal is affected by values both between the microarchitecture-level builings blocks and within each block. Fundamentally, signals in most side channels (e.g. EM and power) are created during  \textit{bit-flips} at the transistor-level~\cite{VANECK1985269,6766222}, i.e. by changes in values at outputs of individual gates, flip-flops, and entire building blocks. Values that are outputs of individual blocks (e.g. the result produced by an ALU or the value read from a data cache) are typically already tracked during cycle-accurate simulation, so we can account for how changes in these values affect the side channel by adjusting the scaling factors according to data-dependent metrics such as the number of bit-flips. Values within a building block (e.g. values of carry bits within an adder or values produced by individual stages within a multiplier), however, are typically not tracked by a cycle-accurate simulator, and doing so precisely would be very costly in terms of simulation time. However, we note that values at outputs of building blocks tend to drive long interconnect wiring, whereas the wiring that carries values within a block tends to be local. This implies that for most analog side channels, e.g. power, EM, etc., values at outputs of building blocks will tend to have a significantly stronger impact of the overall side channel signal, and this is also supported by our preliminary results. Therefore, bit-flips within a building blocks are likely to have an aggregate effect, i.e. the side channel signal is affected mostly when data values at inputs of the block produce large differences in the amount of internal bit-toggling. Rather than add simulation of such intra-block activity to the cycle-accurate simulator, we plan to account for this in the aggregate, by leveraging the ability of our circuit-level simulators to consider symbolic outputs and produce the envelope that describes how much the signal can vary depending on input values and, if such signal variation is significant enough, by generating symbolic expressions for how the aggregate amounts of internal switching depends on the input values of the block.

\subsubsection{Preliminary Results for Thrust 3}
\label{sec:prelim-transform}

Our preliminary results for this thrust are also based on our work on EMSim~\cite{Nader2020}, where we considered two main contributors to the EM side-channel signal contributed by each building block (in EMSim, an entire pipeline stage of the simple 5-stage pipelined processor is treated as a building block). The first group of contributors, which we call \textit{instruction-dependent} contributors, consists of per-use signal contribution of each micro-architectural block, e.g. whether instruction fetch is happening or not (due to a stall) in a particular cycle, whether the register-file is being written in the last stage of the pipeline or not in a particular cycle, etc.

The second group of contributors is \textit{data-dependent} activity, which is created by bit-flips on the long wires, such as address and data lines going to/from the cache, and in registers that hold values of operands. In reality, operand and data values also change the signal in other parts of the processor, e.g. on wiring between gates within an ALU, but for EMSim we only accounted for address, data, and operand values because those are already available during cycle-accurate simulation, whereas accounting for other bit-flips would have required TRL-level or circuit-level modeling (which was not done for EMSim).

For both types of contributors, we assumed that activity creates a signal that has the same overall shape (an exponentially-decaying sinusoid), because we did not use circuit-level models to get a better estimate of the shape. However, the magnitude of that shape is estimated separately,using measured data, for each contributor. For instruction-dependent contributors, i.e. effect on the signal of a specific type of instruction utilizing a specific pipeline stage, this estimation relies on executing microbenchmarks in which a single instruction, surrounded by NOP instructions, is executed, with all of its operands and data values set to zero, to minimize the effect of data-dependent contributors and create a {\tt baseline} level of activity for each type of instruction in each stage of the processor.

\begin{wrapfigure}{l}{0.4\textwidth}
	\centering
	\includegraphics[width=0.4\columnwidth,clip]{figure/amp2.pdf}
	\caption{The signal amplitude for an {\tt ADD} as it progress in the pipeline (while all other instructions are {\tt NOP}). The actual signal is shown in light color (green). Darker color (black) shows the simulated signal when considering each pipeline stage as a separate source (top), and when considering the entire processor as a single source (bottom), and the largest differences between the two are pointed out using red ellipses.}
	\label{fig:amp}
	\vspace{-5mm}
\end{wrapfigure}
Figure~\ref{fig:amp} shows how the measured EM signal (shown in green/light color in both parts of this figure) changes as an {\tt ADD} instruction progresses through the pipeline while all the other instructions are {\tt NOP}s.
%Using Equ.~\ref{equ:em} \cite{Nader2020} and {\tt NOP}$\rightarrow$ {\tt inst}$\rightarrow$ {\tt NOP} instruction sequence, we used our simulator to generate the signal.
%\begin{equation}
%y(t) = \sum_{n=0}^{+\infty} x[n] \sin(\frac{2\pi(t-nT)}{T_0})  e^{-\theta(t-nT)}  \mathtt{u}(t-nT).
%\label{equ:em}
%\end{equation}
%This equation represents two main effects we observed in the analog side-channel signal: 1)
%\textit{switching activity in a processor is synchronized to its clock} and most of the switching happens right after the positive/negative edge of the clock, which implies that activity is not evenly spread over a cycle and a decaying function can be used to model clock activities; 2) \textit{the received signal is also exposed to oscillations with decreasing magnitude}, i.e., small signal variations can be modeled as sinusoidal oscillations.
The top part of Figure~\ref{fig:amp} also shows the corresponding simulated signal, where the contribution of each stage is modeled as an exponentially-decaying sinusoid snippet, each scaled using the magnitude obtained from measured data. To show why individual stages should be modeled separately, the bottom part of Figure~\ref{fig:amp} shows the simulated signal when all stages are treated as being the same, i.e. only one (average) amplitude is estimated from measurements and used for activity in any stage. As can be seen, failing to model each stage individually (as used in previous work~\cite{McCann:2017:TPT:3241189.3241207}) can lead to significant inaccuracies for most stages.

For data-dependent contributors, the pipeline simulator used in EMSim identifies the number of bit-flips on address and data lines and in the register being written. Intuitively, more bit-flips cause more current flow and thus larger amplitude of EM emanations. To combine the effects of such data-dependent activity with the baseline (instruction-dependent) activity, we define the \textit{activity-factor}, $\alpha$, as a \textbf{\textit{scaling factor}} for each baseline activity $A$. To find this $\alpha$, we assume all bit-flips within the stage contribute equally, i.e. that each bit-flip contributes equally to the signal amplitude. Becasue the baseline signal amplitude already includes some bit-flips (e.g. becasue instruction addresses and instructions themselves are non-zero when accessing the instruction cache, we calculate $\alpha$ as $\alpha = 1 + \frac{(\mathit{flips}_{new} - \mathit{flips}_{base})}{\mathit{flips}_{total}}$, where $\mathit{flips}_{new}$ is the total number of flips for the current instruction,  $\mathit{flips}_{base}$ is the total number of flips when previous instruction is {\tt NOP}, and $\mathit{flips}_{total}$ is the maximum possible number of flips for the current instruction. The overall scaling factor for the (decaying sinusoid) snippet is then $A' = \alpha\times A$.

\begin{wrapfigure}{l}{0.4\textwidth}
	\centering
	\includegraphics[width=0.4\columnwidth]{figure/alpha2.pdf}
	\caption{Effect of the \textit{activity factor} on the amplitude. The actual signal shown in green. The simulation is shown in black when activity factor is modeled using a linear regression model (top) and when an \textit{average} activity is used (bottom).}
	\label{fig:alpha}
\end{wrapfigure}
To illustrate why these data-dependent considerations need to be accounted separately for each block, Figure~\ref{fig:alpha} shows the measured signal as green/weak and the simulated signal as black, where the simulated signal in the top part uses per-stage activity factors in each cycle as described above, while in the botom part the simulation always uses one (average) activity factor. We observe that accounting for data-dependent activity in at lest some parts ofthe processor is needed to achieve good accuracy of the simulated signal. Further analysis of measured data has also shown that not all bit-flips have the same effect on the signal. Specifically, we found that 1) bit-flips in the output of the ALU and the cache have the most significant impacts on the signal, which can be expected in this FPGA-based design because these outputs were the driving the longest total amount of interconnect to reach all of their fanout targets, and 2) bit-flips in about 64\% of block output bits had a negligible contribution to overall accuracy (as determined through stepwise regression~\cite{f-test}). This supports our proposed approach of using measured data to estimate the contribution of each block (and how much of it is data-dependent), and then selectively add accounting for data-dependent activity only for blocks where such accounting can result in a meaningful improvement in accuracy. Also, our preliminary findings are encouraging because the largest data-dependent contributions were outputs of entire units (ALU, cache, etc.) rather than internal wiring within these units, i.e. the most impactful bit-flips occur in values that are already modeled by cycle-acurate simulators, while the values on wires within an individual block have less of an effect. However, in our proposed work we plan to use aggregate accounting for toggling within blocks through symbolic circuit-level simulation to obtain not only the ``baseline'' signal snippets for each block, but also the input-dependent expressions for how the shape of the signal snippet would be affected by the values at the block's inputs and, for register files, caches, etc., by changes to values stored within those blocks.

\subsection{Valiidation/Callibration Aginst Measured Signals}

In this thrust we will evaluate the accuracy of snippet generation, transformation, and overall simulated signals produced by the methods we develop, using measured side channel signals collected from real processors. The results of this evaluation will help us identify likely sources of error and refine our methods, which will allow us to refine our approach to create efficient simulators that produce highly accurate side channel signals that are callibrated against specific real processors.

The insights from this evaluation, refinement, and calibration efforts will hen be used to design systematic mechanisms that can automate this iterative refinement/callibration, resulting in mechanisms that selectively refine the circuit-level models, accounting for data-dependent behaviors, scaling factors, and other aspects of the training process to achieve a desired level of accuracy while minimizing the impact of these refinements on simulation time and avoiding costly refinements (e.g. development of more detailed circuit-level models) that would not result in a commensurate increase in accuracy of the simulated signals.

\subsubsection{Preliminary Results for Thrust 4}

Co-PI Kumar in experienced in digital design, chip design,
tape-out, and testing, and runs an undergraudate course in which students design processors that are then are manufactured by TSMC and tested on the UIUC campus.
His group has also participated in Intel's Chip Design Challenge, manufacturing RISC-V microprocessors in Intel's 16 nm technology. This gives us a sizeable collection of microprocessor artifacts, with both large and small microarchitectural differences,
which can be subjected to side-channel measurement and also modeled at various levels of abstraction -- high-level microarchitecture, RTL, gate-level,
GDS/geometry level, and physical chips. This will allows us to evaluate the predictive accuracy of the simulation-generated signals with respect to microarchitectural changes,
and also to calibrate based on numerous real-world designs and implementations.

PI PRvulovic and Co-PI Zajic have extensive experience in side-channel measurements for EM, power, and several other analog side channels (acoustic, optical, temperature, ...), comparing these signals to each other, and, more recently, comparing synthesized or modeled signals to measured ones~\cite{Nader2020} and using the results of these measurements to refine the models. 

\section{Integrated Research, Education, and Outreach Plan}
The research, education, and outreach milestones for each year are outlined in the following table.
\begin{small}
\begin{tabular}{||c|p{0.42\linewidth}|p{0.42\linewidth}||}
\hline
\hline
\textbf{Year} & \textbf{Research} & \textbf{Education and Outreach}\\
\hline
\textit{1} & Develop methods that allow microarchitecture-level simulation to produce highly accurate analog side channel signals; 
Develop methods for efficiently performing circuit-level simulation of caches, functional units, and other building blocks of the processor, to provide per-event timelines for voltage and current; Develop methods that produce per-event signal snippets for various side channels by transforming the outputs of circuit-level simulation.

& Develop initial modules for graduate courses, visits to local schools, undergraduate teams work on lab measurement setups\\
\hline
\textit{2} & Refine methods to allow for per-block and per-code-region breakdowns and attribution to help computer architects and software developers assess the impacts of microarchitectural and software changes; Refine circuit-models to include other metrics that are needed to construct signal snippets for various side channels, along with other useful information such as how much the values of the metrics depend on data values; Develop methods that decompose measured signals.
&
Develop course modules for graduate and undergraduate courses; visit school for hands-on demonstrations, undergraduate teams experiment with refinements to measurement, training, and matching methodology \\
\hline
\textit{3}  &Develop tools to evaluate the tradeoff between performance, power, and side channel leakage; Integrate circuit models into overall architecture side channel simulator; Develop methods that estimate per-block scaling factors for adding these snippets to the overall analog side channel signals; Validate overall side channel simulations against measured side channel signals.
 &
Refine course modules; develop K-12 and public demonstrators for sensory experiences of HT detection (e.g., use software defined radio setups to demonstrate HT detection); fully integrate undergraduates in research\\
\hline
\hline
\end{tabular}
\end{small}

The PIs plan to conduct the proposed research in a highly integrated manner, as the project requires technical and research expertise from several Computer Science (CS) and Electrical Engineering (EE) disciplines, and each PI brings skills essential for the success of this work. To keep the project integrated, we will regularly communicate among the tems members, hold bi-weekly project-wide virtual meetings, and meet in person at least once per year (more details are in our Collaboration Plan).

\subsection{Proposed Outreach and Education Activities}

In addition to research proposed here, this proposed work will help foster interaction among researchers from a
very diverse set of research areas and will involve training students
with a considerable multidisciplinary expertise. To further increase
the impact of this work, we are also planning to perform the following
outreach and education activities.

\begin{itemize}\denseitems

\item We will continue to introduce relevant cross-disciplinary material into several courses at the graduate and undergraduate level.
For example, we will add to VLSI and computer architecture courses an introduction to physical side-channel signals.
Similarly, in electromagnetics, telecommunications, and signal processing classes, we will add a discussion of how computer
hardware and software interact to create side-channel signals. This will help students understand the broader perspectives relevant to
each class, and also help them appreciate the increasingly multi-disciplinary nature of science and technology.

\item We will continue to include undergraduate students in our research.
PIs Zajic and Prvulovic have long history of advising undergraduate students (26 undergrads)
through Opportunity Research Scholar (ORS) program at Georgia Tech and individual mentorship,
while PI Kumar has advised several undergraduate researchers who have gone on to join graduate
studies at UIUC and elsewhere, and has advised undergraduate students who have recently published
in premiere computer architecture conferences, including ISCA, HPCA (best paper nomination),
and DAC. The interdisciplinary nature of the project will help us engage a large number of undergraduate
students in research, and will provide both graduate and undergraduate student with a unique interdiciplinary skill-set.

\item Another outreach activity will be the development of displays
  and tools for educating the general public, especially high-school
  students and teachers, on issues related to both HTs and side-channel signals. This initiative will be
  pursued through Georgia Tech’s established outreach programs for
  high schools in the greater metro Atlanta area~\cite{Conrad2022}.
  The PIs Zajic and Prvulovic have done this in the past with hands-on demonstration for the Family Science Night at the Mimosa
  Elementary School in Roswell, GA and hands-on STEM activities at Pace Academy in Atlanta, GA.
  Similarly, Dr. Kumar has assisted in the organization of events such as HackIllinois and Engineering Open House (EOH)
  that see enthusiastic participation from the local community. He will have a booth on “processors that leak information” in the
  next three EOHs to get high school students excited about intersection of computer architecture and security.

\end{itemize}


\section{Results from Prior NSF Support}
\label{sec:prior}
\noindent
{\bf Dr. Prvulovic and Dr. Zajic} have served as PIs and Co-PIs on the following NSF-funded projects in the last five years CCF-1563991 SHF (from 2016-2022 $\$850K$) and CNS 1740962 (from 2017-2019, $\$199,866$).

Grant CCF-1563991 SHF \textit{Spectral Profiling: Understanding Software Performance without Code Instrumentation}
($\$850K$, October 2013- September 2016).
\textit{Intellectual Merit:}Our seminal work on EM emission based program profiling provided the basis for
future work in the more general area of program analysis techniques that leverage the physical side
effects of computation \cite{Zop, Zop2, Nader2016, Elvan2021}.
\textit{Broader Impact:}This work has opened up new possibilities in a number of additional areas,
including program testing and debugging and software security. Inherently interdisciplinary, research has improved the state of the art in many areas including electromagnetic, signal processing, computer architecture, and software engineering.

Grant CNS 1740962\textit{EAGER: Exploration of THz Backscattering
as a Side-channel in Computer Systems}
($\$107K$, September 2010--August 2012).
\textit{Intellectual Merit:}Our seminal work on backscattered signals and how they can be used for HT detection has opened up new possibilities
for non-destructive testing of electronics and for more precise detection of dormant HTs \cite{Pavel2020,nguyen19,Erik2022}.
\textit{Broader Impact:}THz back-scattering side-channel is an entirely new side channel that significantly differs
from existing ones for both defensive and offensive uses. We have built several testbeds and demos and demonstrated it to various communities
to educate public about this new technology. We have received several best paper and patent awards.

{\bf Dr. Zajic} is currently also the PI on grant ECCS-1651273 NSF CAREER(from 2017-2023 $\$500K$) \textit{Modeling and Measurements for THz Wireless Chip-to-Chip Communications Propagation} \textit{Intellectual Merit:}To enable future THz wireless communication between chips in a system and between blades and racks in large-scale data center systems, this project is investigating  propagation mechanisms \cite{Cheng2020}, \cite{Kim2016}, \cite{Fu2019} and develops channel models \cite{Kim2016a}, \cite{Fu2020} to enable communication between chips on a motherboard inside a computer system, between the motherboard and an add-on card (e.g., a graphics card), between boards/blades in a rack-mounted system typical for base stations, and between racks in a data center environment (with raised floors, rows of racks, cooling ducts, etc.). \textit{Broader Impact:} In addition to the broader impact within the research community and training graduate students funded by this grant, the PI is leading mm-wave indoor channel modeling and measurements efforts in the 5G Millimeter Wave Channel Model Alliance, led by National institute of Standards and Technology, which aims to produce more accurate, consistent and predictive channel models for frequencies above 6 GHz and lead the standardization process \cite{Book}.

Dr. Kumar has served as PI on the following NSF-funded project in
the last five years: CCF-2006763 SHF (from 2020-2023 \$500K).

CCF-2006763 SHF Printed Computer Systems (\$500K, October 2020- September 2023).
\textit{Intellectual Merit:} The project has already resulted in first sub-cent microprocessor~cite{bleier22}, first plastic encryption engines~\cite{bleier23},
first work on printed microprocessors~\cite{bleier2020printed}, and first work on printed classifiers~\cite{mubarik20}.
\textit{Broader Impact:} This project has opened up new possibilities for ultra-low cost and conformal active electronics, enabling applications such as wearable patches~\cite{8502791},
garments~\cite{garments}, packaging~\cite{gethin2013printed} for fast moving consumer goods (FMCG), low-end healthcare products such as smart
bandages~\cite{DERAKHSHANDEH20181259}, and disposable sensors for food~\cite{7994325}, pharmaceuticals~\cite{BERGAMINI200554},
agriculture and forestry~\cite{chemosensors8040125}, and environment~\cite{MARRAZZA1999297} --- that have not seen much penetration of computing.

\bibliographystyle{abbrv}
\bibliography{refs}
\end{document}

