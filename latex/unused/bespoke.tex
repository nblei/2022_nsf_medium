\color{red}
Most software developers lack both the know-how and the equipment to evaluate
their software's potential vulnerability to analog side channels.  As such,
blackbox software which enables developers to analyze their software for analog
side channels is valuable. Analysis of a \textit{program} is difficult
vis-\`a-vis analysis of an \textit{single execution} of the program, because
analyzing the program means analyzing all possible executions of the program.
In fact, program inputs can also effect microarchitectural events (e.g.,
stall due to branch predictor miss on data-dependent branch, cache hit/miss
based on input-dependend data accesses, etc).  It is possible that
one program execution does not result in leakage of sensitive information
via analog side channel, while a different execution of the same program
does result in leakage of sensitive information via analog side channel.
As such, it is important for developers to be able to
analyze \textit{all} possible executions of their programs.

The PIs previous work using symbolic execution of programs to analyze net
switching behavior~\cite{cherupalli2017} and perform taint
tracking~\cite{cherupalli20172} is highly relevant, as it enables capturing
switching activities for all possible executions of a given program.  The prior
work can be modified to meet the abstraction level of the microarchitectural EM
side channel model (Sec.~\ref{sec:preliminary_results}).
Developers will be able to use this symbolic execution framework to perform
microarchitecturally aware analysis of their software's susceptibility to
analog side channel information leakage without any additional expertise or
equipment (in fact, they will not even need the targetted microprocessor).

