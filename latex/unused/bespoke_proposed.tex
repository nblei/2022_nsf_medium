\color{red}
Our symbolic execution framework, used to enable side-chanel aware software
development, currently uses detailed gate-level models of the microprocessor
hardware.  This approach results in high fidelity reproduction of the
microprocessor's EM side-channel, but comes at a high computational cost,
limiting its utility to a software developer.
For example, an FFT implementation for a 16-bit MSP430 DSP takes over 10,000
seconds to run on the gate-level framework.  This means that a developer would
need to wait nearly three hours between modifying the program, and getting
results for its impact on EM side-channel information leakage.  We will raise
the abstraction level of our symbolic execution framework from the gate-level
to the MISO model abstraction detailed in Sec.~\ref{sec:preliminary_results}.
There are several pecularities which make this an interesting task.  First,
data-dependent activities will often be evaluated for symbolic data.  As such,
for a given time period, the EM signal is best modeled as a distribution.
Second, microarchitectural events (i.e., stalls, cache misses, and branch
mispredictions) will also be dependent on symbolic data (e.g., a symbolic
address resulting in either a cache hit or a cache miss).  Thus, great care
must be taken to ensure the path explosion problem is not exascerbated by the
modeling of microarchitectural events.  These pecularities mean that performing
symboilc execution at the MISO model level is not straightforward,
hence enabling side-channel aware software development requires further
research effort.
Additionally, identifying relevant information flow policies and determining
the EM characteristics associated with violations of the policies requires
further effort.
\color{black}


