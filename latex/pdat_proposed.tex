\color{red}
We will create a version of PDAT purpose built for identifying potential EM
side-channel leakage using the high-level MIMO model, rather than a detailed
netlist model. Unlike symbolic execution, model checking does not suffer from
the path-explosion problem (model checking's computational complexity is based
on the size of the model's state space), thus handling microarchitectural
events poses no significant difficulties. Since the high-level model has
significantly smaller state-space than a microprocessor's netlist, this will
lead to significant decrease in model checking runtime.
It will also enable computer architects to
rapidly explore the architectural design space for EM side-channel behavior.
For example, an architect can determine the impact of branch delay
slots~\cite{lalja88} on side-channel behavior by disabling the branch
misprediction event within PDAT, or determine the impact of software based
multiplication by restricting the execution environment to non-multiply
instructions.

PDAT's environment restrictions allows us to also use PDAT for
software-hardware co-analysis.  We can restrict the environment to a single
program, and use model checking to explore all possible executions of the
program for a given hardware model.  This enables hardware-software codesign
flows to be EM side-channel aware.

A model checking approach can also be used purely by software developers, as an
alternative to the symbolic execution approach, by restricting the execution
environment to a single program.  As the performance of model checking and
symbolic execution are both dependent on the particular MISO model as well as
the particular program, it is likely that in some cases, symbolic execution
will perform better than model checking, and vice versa.  This makes both
approaches to side-channel aware software development valuable.
\color{black}


